\section{Introduction}
%
This article follows several papers and memoirs on the \linbox%
%
\footnote{See \url{http://www.linalg.org}.}
%
(\cf{} \cite{Giorgi:2004:these,Turner:2002:these,Boyer:2012:these,Dumas:2002:icms,Dumas:2010:lbpar})
and builds upon them.

\linbox is a \cpp template library for fast and exact linear algebra. It is designed with genericity
and efficiency in mind.
% The main concepts include:
% \begin{itemize}[---]
	% \item A hierarchy fields--containers--algorithms--solutions
	% \item Matix containers are BlackBoxes
	% \item RAII reentrant style
% \end{itemize}
%
% \par
%
The \linbox library is under constant evolution, driven by new problems and
algorithms, by new computing paradigms, new compilers and architectures. This
poses many new challenges. To address this changes, we are incrementally
updating the \emph{design} of the library towards a \textsf{2.0} release.
%
\par
%
Let's start from a basic consideration: we show in the \Cref{tab:sloc} the
increase in the size%
%
\footnote{Using \textsf{sloccount}, available at
\url{http://sourceforge.net/projects/sloccount/}.}
%
of \linbox and its dependancies in terms of ``lines of code''.
%
% \par
%
\begin{table}%[htbp]
	\begin{center}
	\vspace{-1em}
	\small
		\renewcommand{\arraystretch}{0.9}
	\begin{tabular}{lcccccccccc}
		\toprule
		\linbox & \sf 1.0.0\td \tdd & \sf 1.1.0\td \tdd& \sf 1.1.6\tdd & \sf 1.1.7\tdd & \sf 1.2.0 & \sf 1.2.2 & \sf 1.3.0 & \sf 1.4.0\\
		loc ($\times \num{1000}$) & {\num{77.3}}& {\num{85.8}} & {\num{93.5}} & {\num{103}} & {\num{108}}  & {\num{109}} &        {\num{112}} & \num{135} \\
		\cmidrule(lr){1-9}
		\fflasffpack &n/a&n/a& n/a & \sf 1.3.3 & \sf 1.4.0 & \sf 1.4.3 & \sf 1.5.0 & \sf 1.8.0 \\
		loc & --- & ---& --- &\num{11.6} & {\num{23.9}} & {\num{25.2}} & {\num{25.5}} & \num{32.1}\\
		\cmidrule(lr){1-9}
		\givaro        & n/a& n/a & \sf 3.2.16        & \sf 3.3.3         &  \sf 3.4.3         & \sf 3.5.0         & \sf 3.6.0 & \sf 3.8.0 \\
		loc&---&---& \num{30.8} & {\num{33.6}}   & {\num{39.4}} & {\num{41.1}} & {\num{41.4}} &  \num{42.8} \\
		\cmidrule(lr){1-9}
		total  &  \num{77.3} & \num{85.8} & \num{124} & \num{137} & \num{171} & \num{175} & \num{179} & \num{210} \\
		\bottomrule \\
	\end{tabular}
	\caption{Evolution of the number of lines of code (loc, in thousands) in
		\linbox, \fflasffpack and \givaro (\td contains \givaro, \tdd
	contains \fflasffpack).}
	\label{tab:sloc}
	\vspace{-3em}
\end{center}
\end{table}%

%
This increase affects the library in several ways.  First, it demands a
stricter developpement model, and we are going to list some techniques we used.
%% SPLITING
For instance, we have transformed \fflasffpack %
%
\footnote{See \url{http://www.linalg.org/projects/fflas-ffpack/}.}
	%
(\cf{} \cite{Dumas:2008:Flas}) into a new stand-alone header library, resulting
in more visibility for the \fflasffpack project (Singular ?) but also in a better
structuration an maintanability of the library, focusing the developpement
areas more precisely.
%
%% FIXING
%
Also, a larger template library is harder to manage, there is more difficulty
to trace, debug and write new code: techniques employed
for easier developpement include reducing
compile times, enforcing stricter warnings and checks, supporting for more
compilers and more architectures, simplifiying and automatising version number
changes, automatising memory leak checks, setting up buildbots to check the code
frequently,\dots
% This demand on the developper
% side is also driven by the fact that code introduced by various one-timers
% needs to be maintained.
%
\par
%
But this increase also forces the library to be more user friendly. For
instance, we have: Developped an \texttt{auto-install.sh} script that installs
automatically the lastest stable or developpement versions of the trio;
Facilitated the discovery of the \textsc{Blas}/\textsc{Lapack} libraries;
Simplified and sped up the checking process while covering more of the library (\danger dave ?);
Added comprehensive benchmarking tools,\dots
%
%
\par
%
Developping generic and high-performance libraries is difficult. We can find a
large litterature on coding standards and software design references in (\cf{}
\cite{alexandrescu:01:modern,gamma:95:design,sutter:05:cpp,stroustrup1994design,Douglas:05:GPHP}),
and many internet sources and a lot of experience acquired by/from free
software projects.
%
\par
%
We are going to describe the advancement in the design of \linbox in the next
three sections. We will first describe the new \emph{container} framework in
\Cref{sec:container}, then improve the \emph{matrix multiplication} algorithms
in \Cref{sec:matmul} by contributing special purpose matrix multiplication
plugings, and finally present the new \emph{benchmark/optimisation}
architecture (\Cref{sec:bench}).
%
\danger develop this § more later
