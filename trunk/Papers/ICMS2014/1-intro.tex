\section{Introduction}
%
This article follows several papers and memoirs concerning \linbox%
%
\footnote{See \url{http://www.linalg.org}.}
%
(\cf{} \cite{Giorgi:2004:these,Turner:2002:these,Boyer:2012:these,Dumas:2002:icms,Dumas:2010:lbpar})
and builds upon them.
%
\linbox is a \cpp template library for fast and exact linear algebra, designed with generality
and efficiency in mind.
% The main concepts include:
% \begin{itemize}[---]
	% \item A hierarchy fields--containers--algorithms--solutions
	% \item Matrix containers are Blackboxes
	% \item RAII reentrant style
% \end{itemize}
%
% \par
%
The \linbox library is under constant evolution, driven by new problems and
algorithms, by new computing paradigms, new compilers and architectures. This
poses many challenges: we are incrementally
updating the \emph{design} of the library towards a \textsf{2.0} release.
The evolution is also motivated by developing a high-performance mathematical
library available for researchers and engineers that is easy to use and help
produce quality reliable results and quality research papers.
%
\par
%
Let us start from a basic consideration: we show in the \Cref{tab:sloc} the
increase in the ``lines of code'' size\footnote{Using \textsf{sloccount}, available at
\url{http://sourceforge.net/projects/sloccount/}.}
%
of \linbox and its coevolved dependencies \givaro and
\fflasffpack\footnote{symbol \td when \givaro is included and \tdd when
contains \fflasffpack}.
%
% \par
%
\begin{table}%[htbp]
	\begin{center}
	\vspace{-1em}
	\small
		\renewcommand{\arraystretch}{0.9}
	\begin{tabular}{lcccccccccc}
		\toprule
		\linbox & \sf 1.0.0\td \tdd & \sf 1.1.0\td \tdd& \sf 1.1.6\tdd & \sf 1.1.7\tdd & \sf 1.2.0 & \sf 1.2.2 & \sf 1.3.0 & \sf 1.4.0\\
		loc ($\times \num{1000}$) & {\num{77.3}}& {\num{85.8}} & {\num{93.5}} & {\num{103}} & {\num{108}}  & {\num{109}} &        {\num{112}} & \num{135} \\
		\cmidrule(lr){1-9}
		\fflasffpack &n/a&n/a& n/a & \sf 1.3.3 & \sf 1.4.0 & \sf 1.4.3 & \sf 1.5.0 & \sf 1.8.0 \\
		loc & --- & ---& --- &\num{11.6} & {\num{23.9}} & {\num{25.2}} & {\num{25.5}} & \num{32.1}\\
		\cmidrule(lr){1-9}
		\givaro        & n/a& n/a & \sf 3.2.16        & \sf 3.3.3         &  \sf 3.4.3         & \sf 3.5.0         & \sf 3.6.0 & \sf 3.8.0 \\
		loc&---&---& \num{30.8} & {\num{33.6}}   & {\num{39.4}} & {\num{41.1}} & {\num{41.4}} &  \num{42.8} \\
		\cmidrule(lr){1-9}
		total  &  \num{77.3} & \num{85.8} & \num{124} & \num{137} & \num{171} & \num{175} & \num{179} & \num{210} \\
		\bottomrule \\
	\end{tabular}
	\caption{Evolution of the number of lines of code (loc, in thousands) in
		\linbox, \fflasffpack and \givaro (\td contains \givaro, \tdd
	contains \fflasffpack).}
	\label{tab:sloc}
	\vspace{-3em}
\end{center}
\end{table}%

%
This increase affects the library in several ways.  First, it demands a
stricter development model, and we are going to list some techniques we used.
%% SPLITTING
For instance, we have transformed \fflasffpack %
%
% \footnote{See \url{http://www.linalg.org/projects/fflas-ffpack/}.}
	%
(\cf{} \cite{Dumas:2008:Flas}) into a new standalone header library, resulting
in more visibility for the \fflasffpack project
% (\danger Singular ?)
and also in better structure and maintainability of the library.
% focusing the development areas more precisely.
%
%% FIXING
%
A larger template library is harder to manage. There is more difficulty
to trace, debug, and write new code. Techniques employed
for easier development include reducing
compile times, enforcing stricter warnings and checks, supporting more
compilers and architectures, simplifying and automating version number
changes, automating memory leak checks, and setting up buildbots to check the code
frequently.
% This demand on the developer
% side is also driven by the fact that code introduced by various one-timers
% needs to be maintained.
%
\par
%
This size increase also requires more efforts to make the library user friendly. For
instance, we have:
%
Developed %an \texttt{auto-install.sh}
scripts that install automatically the
latest stable/development versions of the trio, resolving version
dependencies;
%
Eased the discovery of \scsf{Blas}/\scsf{Lapack} libraries;
%
Simplified and sped up the checking process, covering more of the library;
% (\danger dave a word on make fullcheck or on the future feature matrix ---what
% solution, what field, what implementation we recommend/provide if at all ?);
%
Updated the documentation and distinguished user and developer oriented docs;
%
Added comprehensive benchmarking tools.
%
%
\par
%
Developing generic high performance libraries is difficult. We can find a
large literature on coding standards and software design references in (\cf{}
\cite{alexandrescu:01:modern,gamma:95:design,sutter:05:cpp,stroustrup1994design,Douglas:05:GPHP}),
and draw from many internet sources and experience acquired by/from free
software projects.
%
% \par
%
We describe advances in the design of \linbox in the next
three sections. We will first describe the new \emph{container} framework in
\Cref{sec:container}, then,
in \Cref{sec:matmul},
the improved \emph{matrix multiplication} algorithms
made by contributing special purpose matrix multiplication plugins, and, finally, we present the new \emph{benchmark/optimization}
architecture (\Cref{sec:bench}).
%
\par
% (\danger develop this § more later when I have more material.)
