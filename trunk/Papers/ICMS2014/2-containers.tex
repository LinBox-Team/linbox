\section{Containers architecture}\label{sec:container}
%
\linbox is mainly conceived around the RAII (Resource Acquisition Is
Initialization, see \cite{stroustrup1994design}) concept with reentrant
function. We also follow the {founding scope allocation} model (or \emph{mother
model}) of \cite{Dumas:2010:lbpar} which ensures that the memory used by
objects is allocated in the constructor and freed only at its destruction. The
management of the memory allocated by an object is exclusively reserved to it.
%
\par
%
\linbox uses a variety of container types (representations) for matrix and vectors over fields and rings.
The fragmentation of the containers into various matrix and
blackbox types has been addressed and simplified. The many different matrix and
vector types with different interfaces has been reduced into only two
%(possibly essentially one in the future)
containers: \texttt{Matrix} and
\texttt{Vector}.
%
\subsection{General Interface for %Vectors and
Matrices}
%
First, in order to allow operations on its elements, a container is
parameterized by a field object (\Cref{code:clmat}), not the field's element
type. This is simpler and more general.
Indeed, the field element type can be inferred from a
\verb!value_type! type definition within the field type.
Then, the storage type is given by a second
template parameter that can use defaults,
\eg dense \blas matrices (stride
and leading dimension or increment), or some sparse format.
% what else can it be? It can default to other dense formats such as vectors of
% rows or anything.
%
{
	\small
\begin{lstlisting}
template< class _Field, class _Storage = denseDefault >
class Vector ;
\end{lstlisting}
\lstcapt{Matrix and vector classes in \linbox.\label{code:clmat}}
}
% template< class _Field, class _Storage = denseDefault >
% class Matrix ;

%
In the founding scope allocation model, we must distinguish containers that own (responsible for
dynamically allocated memory) and containers that share memory of another.
\texttt{SubMatrix} and \texttt{SubVector} types share the memory;
\texttt{Matrix} and \texttt{Vector} own it.
%
All matrix containers share the common \applin  interface
described in the next paragraphs, it accommodates both owner and sharer
container types, and defines the minimal methods required for a
template \applin matrix type:
% \Cref{code:bb}.
%
% {
	\small
\begin{lstlisting}
template< ¬\ldots¬ >
class Matrix {
public:
        /* constructors */
	// no empty constructor
	// with a field, rows, columns, other matrix\dots

	/* y <- A.x or y <- A^T.x */
        template<class _in, class _out>
        _out& apply(_out &y, const _in &x) const;
        template<class _in, class _out>
        _out& applyTranspose(_out &y, const _in &x) const;

        /* operator rebind */
        template<typename _Tp1>
        struct rebind ;

        /* dimensions */
        size_t rowdim() const;
        size_t coldim() const;

        /* field */
        const Field& field() const;

	/* conversions */
	// resize
	// export/import to default types

	/* building the matrix */
	// read from MM/stdout/...
	// write to MM/stdout/...
	// setEntry/init()/finalise()/optimize()
protected:
        /* internals */
        ¬\ldots¬
};
\end{lstlisting}
\lstcapt{Interface of an \applin\label{code:bb}}
}
	% ¬\ldots¬
	% /* types */
	% // submatrix types
	% ¬\ldots¬

%
% \begin{description}
%
	% \item[Input/Output.]
\def\monitem#1{\par\textit{#1}\ }
\monitem{Input/Output.}
		Our matrix containers all read and write from Matrix Market
		format\footnote{See \url{http://math.nist.gov/MatrixMarket/}.}
		which is well established in the numerical linear algebra
		community and facilitates sharing matrices with other software
		tools.  The MatrixMarket header comment provides space for
		metadata about the provenance of a matrix and our interest in
		it.  However, because of our many entry domains and matrix
		representations, extensions are necessary to the MatrixMarket
		format.  For instance, the header comment records the modulus
		and irreducible polynomial defining the representation of a
		matrix over GF($p^e$).
% we haven't committed to this as a MM variant, should we? bb: yes!..
		We can further adapt the header to suit our needs, for instance
		create new file formats that save space (\eg CSR fashion saves
		roughly a third space over COO, \cf Harwell-Boeing format).
		Structured matrices (Toeplitz, Vandermonde, \etc) can have file
		representations specified.
%
\begin{comment}
% setEntry is not universal to all blackboxes so skip for abstract
	% \item[Accessing Elements.]
		\monitem{Accessing Elements.}
		The function \texttt{setEntry} can be
		used to populate the matrix (from some \texttt{init()} until a
		\texttt{finish()} is emitted).  The function \texttt{setEntry}
		can be costly (for some sparse formats for instance)
		% (\danger Dave, more words on the init/finish?)
% so set entry should always build some COO or CSR and then
% convert to the format.
%
		Other functions such as {\tt refEntry} (that retrieves a
		writable reference to an entry) may be difficult to implement
		or inefficient (compressed fields, sparse matrices);
		\texttt{getEntry} may be specialized.
		% , in all cases, there is a
		% solution for this operation falling back to the {\tt apply}
		% method.
		% The function {\tt clearEntry} may be used to zero an entry and
		% update the storage.
		% (\danger setEntry to zero is sometimes
		% buggy, is this a feature?)
		%
		% \par
		%
		Accessing (read or write) elements, rows or columns through
		iterators, although handy, is not required because it can be
		tricky to implement (compressed fields, hybrid or
		recursive formats). Traits on matrices where efficient iterators
		are available allow for their selections in some algorithms.
\end{comment}
		\begin{comment}
		(\danger We cannot have {\tt const} iterators for all formats
		(see hybrid), walking through a matrix in unspecified order can
		be problematic, for instance, in an hom when a zero may
		alter/change the representation. I don't understand hom more
		than I understand init/rebind by the way, and I won't
		understand it until the morphism is given/chooseable/enforced.
		Rebind is really hom, so why is it in the interface anyway?
		This is the same with getEntry. We don't require it because we
		have a solution. Also, if apply is to be implemented via a
		domain or {\tt mul}, we could stress it is in the interface for
		convenience when writing code only. End of \danger)
		\end{comment}
%
	% \item[Apply method.]
		\monitem{Apply method.}
		This is essential in the \applin interface (\Cref{ssec:apply,sec:matmul}).
%
	% \item[Rebind/Conversions.]
		\monitem{Rebind/Conversions.}
		In addition to the rebind mechanism (convert from one field to the other), we add
%
	 conversion mechanisms between formats, for instance all
	 sparse matrix formats can convert to/from CSR format:  this `star'
	 mechanism can simplify the code (to the expense of memory usage) and
	 may speed it up when some central formats  are
	 well tuned for some task.
	 % Such tasks could be
	 % read/write from a file for instance, or update the matrix.
%
% \end{description}
	 % \paragraph{}
	 \par
%
This is a common minimal interface to all our matrix containers that can be
used by all algorithms.
% Additional elements to a container can be added, and flagged
% with a trait.
This interface provides the basic {\em external}
functionality of a matrix as a
 ``linear mapping'' (black box).
This interface is shared by: \emph{dense}
containers (\blas-like,\ldots); \emph{permutation} containers (compressed
\scsf{Lapack} or cycle representation); \emph{sparse} containers (based on
common formats or on STL containers such as {\tt map}, {\tt deque},\ldots);
\emph{structured} containers ({\tt Diagonal}, {\tt Hankel}, {\tt Butterfly},\ldots);
\emph{compound} containers ({\tt Compose}, {\tt Submatrix},\ldots).
Additional functions of a container can be added, and flagged with a trait, for
example those that support internal changes as for Gaussian elimination.
%
\subsection{The \texttt{apply} method}\label{ssec:apply}
%
% We will discuss the \texttt{apply} function in \Cref{sec:apply}.
% Both vec/mat
%
\par
%
The \texttt{apply} method (left or right) is arguably the most important
feature in the matrix interface and the \linbox library. It performs what a
linear application is defined for: apply to a vector (and by extension  a block
of vectors, \ie a matrix).
%
\par
%
We propose the new interface (\Cref{code:apply}), where {\tt \_In} and {\tt
\_Out} are vector or matrices, and {\tt Side} is {\tt Tag::Right} or
{\tt Tag::Left}, whether the operation $y \gets A^{\top} x$ or  $y \gets A x$ is
performed. We also generalize to the operation $y \gets \alpha A x + \beta y$.
% as it is often handy and efficient to allow for accumulation.
%
{
\small
\begin{lstlisting}
// y = A.x
template< class _In, class _Out  >
_Out& apply(_Out &y, const _In& x, enum Side) ;

// y = alpha y + beta A.x
template< class _In, class _Out  >
_Out& applyAcc(_Out &y, const Element& alpha, const _In& x, const Element& beta, enum Side) ;

\end{lstlisting}
\lstcapt{Apply methods.\label{code:apply}}
}

%
This method is fundamental as it is the building block of the \applin
algorithms (for instance block-Wiedemann) and as the matrix
multiplication, main operation in linear algebra, needs to be extremely efficient (\Cref{sec:matmul}).
%
%
% \par
%
The implementation of the apply method can be left to a \mul solution,
which can include a helper/method argument if the apply
parameters are specialized enough.
%
% \Cref{code:applymul}.
%
% Usually,  the {\tt apply} method for
% containers is strongly dependent on the representation of the matrix. In
% the case of Hybrid containers, it will simply be calls to the {\tt apply}
% method of the children.
%
