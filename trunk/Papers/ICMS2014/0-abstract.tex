\begin{abstract}
	%
	We develop in this paper design techniques used in the \cpp exact
	linear algebra library \linbox. They are intended to make the library
	safer and easier to use, while keeping it generic and efficient.
	%
	\par
	%
	First, we review the new simplified structure of the containers, based
	on our \emph{founding scope allocation} model.
	% Namely, vectors and matrix containers are all templated by a field
	% and a storage type.
	% Matrix interfaces all agree with the same minimal black-box interface.
	This allows e.g. for a unification of our dense and sparse matrices, as
	well as a clearer model for matrices and submatrices. We explain the
	design choices and their impact on coding. We will describe serveral of
	the new containers, especially our sparse and dense matrices storages
	as well as their \apply (\emph{black-box}) method and compare to
	previous implementations.%
	\par
	%
	Then we present a variation of the \emph{strategy} design pattern that
	is comprised of a controller--plugin system: the controller (solution)
	chooses among plug-ins (algorithms) and the plug-ins always call back
	the solution so a new choice can be made by the controller. We give
	examples using the solution \mul, and generalise this design pattern to
	the library.
	% We also show performance comparisons with former \linbox versions.
	%
	\par
	%
	Finally we present a benchmark architecture that serves two purposes.
	The first one consists in providing the user with an easy way to
	produces graphs using \cpp. The second goal is to create a framework
	for automatically tuning the library (determine thresholds, choose
	algorithms) and provide a regression testing scheme.
	%
        \keywords{\linbox; design pattern; algorithms and containers; benchmarking;
	matrix multiplication algorithms; exact linear algebra.}
\end{abstract}
