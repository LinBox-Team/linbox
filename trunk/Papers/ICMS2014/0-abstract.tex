\begin{abstract}
	%
	We develop in this paper design techniques used in the \cpp exact
	linear algebra library \linbox, intended to make the library
	safer and easier to use, while keeping it generic and efficient.
	%
	% \par
	% \\
	%
	First, we review the new simplified structure for containers, based
	on our \emph{founding scope allocation} model: providing \eg
	unification for our matrices classes, a clearer model for matrices and
	submatrices. We explain design choices and their impact on coding.
	%
	% \par
	% \\
	%
	Then we present a variation of the \emph{strategy} design pattern that
	is comprised of a controller--plugin system: the controller (solution)
	chooses among plug-ins (algorithms) and the plug-ins always call back
	the controller. We give examples using the solution \mul.
	%
	% \par
	% \\
	%
	Finally we present a benchmark architecture that serves two purposes:
	Providing the user with easier ways to produce graphs using
	\cpp; Creating a framework for automatically tuning the library
	(determine thresholds, choose algorithms) and provide regression
	testing schemes.
	%
        \keywords{\linbox; design pattern; algorithms and containers; benchmarking;
	matrix multiplication algorithms; exact linear algebra.}
\end{abstract}
