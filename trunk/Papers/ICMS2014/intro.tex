\section{Introduction}

The \linbox library is under constant evolution, driven by new problems and
algorithms, new compilers, new computing paradigms. This poses new challenges.
We are incrementally updating the design of the library towards a \textsf{2.0}
release.
%
\par
%
We show in the next \cref{tab:sloc} the increase in the size%
%
\footnote{Using \textsf{sloccount}, available at
\url{http://sourceforge.net/projects/sloccount/}.}
%
of \linbox.
%
\par
%
\begin{table}[htbp]
	\centering
	\caption[Evolution of the number of lines of codes in \linbox]{Evolution of the number of lines of code (loc, in thousands)
		in \linbox, \fflas and \givaro (\textdagger contains \givaro,
	\textdaggerdbl contains \fflaflas).}
	\label{tab:sloc}
	\begin{tabular}{rcccccccccc}
		\toprule
		\linbox & \sf 1.0.0\textsuperscript{\textdagger,\textdaggerdbl} & \sf 1.1.0\textsuperscript{\textdagger,\textdaggerdbl}& \sf 1.1.6\textsuperscript{\textdaggerdbl} & \sf 1.1.7\textsuperscript{\textdaggerdbl} & \sf 1.2.0 & \sf 1.2.2 & \sf 1.3.0 & \sf 1.4.0\\
		loc & {\numprint{77.3}}& {\numprint{85.8}} & {\numprint{93.5}} & {\numprint{103.1}} & {\numprint{107.7}}  & {\numprint{109}} &        {\numprint{111,8}} & xxx \\
		\midrule
		\fflaflas &N/A&N/A& N/A & \sf 1.3.3 & \sf 1.4.0 & \sf 1.4.3 & \sf 1.5.0 & \sf 1.x.x \\
		loc & --- & ---& --- &\numprint{11.6} & {\numprint{23.9}} & {\numprint{25.2}} & {\numprint{25.5}} & xxx\\
		\midrule
		\givaro        & N/A& N/A & \sf 3.2.16        & \sf 3.3.3         &  \sf 3.4.3         & \sf 3.5.0         & \sf 3.6.0 & \sf 3.x.x \\
		loc&---&---& \numprint{30.8} & {\numprint{33.5}}   & {\numprint{39.4}} & {\numprint{48.3}} & {\numprint{48.6}} & xxx \\
		\midrule
		total  &  \numprint{77.3} & \numprint{85.8} & \numprint{124} & \numprint{137} & \numprint{171} & \numprint{182} & \numprint{186} & xxx \\
		\bottomrule
	\end{tabular}
\end{table}
%
The increase in the library size demands a stricter developpement model. For
instance, we have put \fflas files into a new stand-alone \fflas header
library, reduced compile times, enforced stricter warnings and checks, support
for more compilers and architectures, simplified and automatised version number
changes, automatised memory leak checks,\dots This demand on the developper
side is also driven by the fact that code introduced by various one-timers
needs to be maintained.
%
\par
%
But this increase also forces the library to be more user friendly. For
instance, we have: Developped an \texttt{auto-install.sh} script that installs
automatically the lastest stable or developpement versions of the trio;
Facilitated the discovery of the \textsc{Blas}/\textsc{Lapack} libraries;
Simplified and sped up the checking process; Added comprehensive benchmarking tools,\dots
%
\par
%
This article follows several papers and memoirs on \linbox
(\cite{Giorgi:2004:these,Turner:2002:these,Boyer:12,Dumas:2002:icms,Dumas:2010:lbpar})
and builds upon them.
%
\par
%
Developping generic and high-performance libraries is difficult. We can find a
large litterature on coding standards and software desin references in
(\cite{alexandrescu:01:modern, gamma:95:design, sutter:05:cpp,
stroustrup1994design,Douglas:05:GPHP}),and  many internet sources and a lot of
experience acquired by free software projects.
%
\par
%
We are going to describe the advancement in the design of \linbox in the next
three sections. We will first describe the new \emph{container} framework in
\cref{sec:container}, then improve the \emph{matrix multiplication} algorithms
in \cref{sec:matmul} by contributing special purpose matrix multiplication
plugings, and finally present the new \emph{benchmark/optimisation}
architecture (\cref{sec:bench}).

