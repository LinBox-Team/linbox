% I write this just to verify the basic facts for myself.  -bds

Theorem.  Let A be a symmetric matrix over a subring of Q.
Then TFAE:

\begin{enumerate}
\item
A is positive definite, i.e., \all x, x^T A x > 0.
\item
All eigenvalues of A are positive, i.e., \all \lambda, x,
A x = \lambda x => \lambda > 0.
\item
All principal minors of A are positive.
\item
All leading principal minors of A are positive.
\item 
A has a Cholesky decomposition, A = V^T V, for upper triangular V. 

Proof: 
1 implies 2:  trivial to show not 2 implies not 1.

begin unfinished:  Go to Gantmacher for a nice treatment!
2 implies 1:  Let a basis B of eigenvectors be given.  In this 
basis  x = \sum x_i b_i and x^T A x = (\sum x_i b_i^T) (\sum \lambda_j x_j b_j).
= 
2 implies 3:  Let B' be a non-positive principal minor on index set I of size i.
let P be a permutation that moves I to the initial i positions.
Then P^T A P has the same eigenvalues and principal minors as A and has 
a non-positive leading principal i by i minor B.  
Then B has a zero or negative eigenvalue with eigenvector v.  Writing
in block form, 
(B  C) (v) = (\lambda v + Cx)
(C' D) (x)   (C' v + Dx     )
End unfinished.

3 implies 4:  trivial.

4 implies 5:  Do A = LU in which U is unit upper triangular and lower 
triangular L has diagonal entries, quotients of leading principal minors, 
are positive.  Let L = M D, in which D is diagonal and M is unit lower 
triangular. Easily, using symmetry of A, M = U^T.
Then we have A = U^T D U, and finally A = V^T V, for V = D^{1/2)U.

5 implies 1:  trivial. 
