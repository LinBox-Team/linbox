\subsection{Fast Rank Certificate}
Sometimes the desired information about a symmetric matrix may just be a determination whether or
not it is positive definite.  The exact signature may not be necessary.  We do not know a way
to determine definiteness that is asymptotically faster than finding the signature.  However,
if the given symmetric matrix is known to be positive (or negative) semi-definite, then we have 
a faster way to distinguish the singular from the non-singular (hence semi-definite from definite).
The certified rank methods of \cite{ssv03} are not asymptotically faster than signature
computation.  Here we give a modification of one of the certified rank algorithms which saves
a factor of $n$.

The fast blackbox algorithm of \cite{Wied86} for computing minimal polynomial mod a prime 
is probabilistic and may fail to compute the true minimal polynomial of the given matrix.
However, with certainty the returned polynomial is a factor of the true minimal polynomial,
and the probability of a proper factor is small.
This algorithm is designed to exploit this property to make a singular/nonsingular determination
with certainty (LasVegas).

\begin{algorithm}{Rank}
\Inspec $A$, an integer $n\times n$ matrix, and\\
\> $S$, a set of nonzero integers.\\
\Outspec $r$, the rank of of $A$, or ``fail".\\

\Stmt[1.] [ Precondition. ]\\
Let D be a diagonal matrix with entries chosen uniformly at random from $S$.\\
Form $B := DA^T AD$.\\
\Stmt[2.] [ Compute $\trace(B)$ over the integers. ] \\
For $i$ from 1 to $n$, Compute $B_{i,i} = i$-th element of $Be_i$, where $e_i$ is the $i$-th column of the identity.\\
Let $T = sum B_{i,i}$

\Stmt[3.] [ Minpoly. ]\\
Compute $m(x) = \minpoly(B) \mod P$, where $P$ is a prime, $|T| < P < 2|T|$.
%[ Using probabilistic algorithm for \minpoly], which is certain to return a factor
%of the true minpoly, 
\Stmt[4.] \lb Result. \rb\\
If $\coeff(m, \deg(m) - 1) \neq t$, return ``fail".\\
If $\coeff(m, 0) = 0, r := \deg(m) - 1$.\\
Otherwise, $r := n$. [ $n = \deg(m)$ in this case. ].\\
Return $r$.
\end{algorithm}

\begin{theorem}
Let an integer $n\times n$ matrix $A$ be given, and a finite set $S$ of nonzero integers.
Let $s$ be the 
cardinality $S$ and 
let $K$ be an upper bound for the absolute values of the entries of $A$ and of $S$,
and let $E$ be the number of integer arithmetic operations 
needed to apply $A$ to a vector, on the left or right.
about $A$.  Algorithm Rank returns ``fail" with probability less than $n^2/s$ (FIXME),
and correctly returns the rank otherwise..
It runs in time $\softO(n(E+n)d).  Thus if $A$ is dense and $d$ constant, the run time
is $\softO(n^3)$ and if $A$ is quite sparse, $E \in \softO(n)$, and $d \in \bO(log(n))$,
the run time is $\softO(n^2)$.
Also, if $S$ is large enough to make the probability of
error less than $1/2$, one obtains a LasVegas algorithm of the same 
asymptotic time by repeating Rank until it succeeds.
\end{theorem}

Proof:
First we consider correctness.
The matrix $B$ constructed in step 1 is positive semidefinite and of the same rank as $A$.
Because of the symmetry it is diagonalizable (eigenvalues are simple).
By \cite{precond} the probability that the nonzero eigenvalues of $B$ are distinct is greater than FIXME.
If so, the characteristic polynomial of $B$ is a power of $x$ times the true minimal polynomial.
Let $m(x)$ be a factor of the minimal polynomial hence of the characteristic polynomial of $B$. 
It's second coefficient
is the sum of it's nonzero roots, namely a sum of a subset of the nonzero eigenvalues of $B$.
But all of the nonzero eigenvalues of $B$ are positive.  Thus the second coefficient of 
$m$ is no more than the trace and equal to it only if $m$ is essentially
the characteristic polynomial.  (Let us say two polynomials are essentially the same
if their quotient is a power or $x$.)
The problem is that we don't know if m is an image of a factor of charpoly over the integers,
so this argument fails.

The entries  of $A^T A$ are of size bounded by $nK^2$, and those of $B = DA^T AD$ are
bounded by $nK^4$.  Hence the trace, sum of the diagonal entries, is bounded by
$n^2 K^4$.  
This value is also a bound for the modulus $P$ used in step 3.
Let $d = \ceil \log(K)$, it follows that basic arithmetic operations mod $P$ and 
the basic integer arithmetic operations on values less than $n^2K^$ cost $\softO(d)$.
(\cite{vzg} contains a thorough discussion of integer arithmetic cost.)
Then application of A to a vector costs $\softO(Ed)$ and, the computation of the 
$n$ diagonal entries and their sum, the trace, costs $\softO(nEd)$.

Now consider step 3.
The minpoly computation requires $2n$ matrix vector products and an additional
$n^2$ operations mod $M$.  This costs $\softO(n(E + n)d)$.
\QED
