Let us say that a vector $v$, such as the
characteristic polynomial coefficients or
leading principal minors with alternative sign,
is a {\signature-revealing vector} for $A$ if $\signature(A) = \signature(v)$.
The algorithms we
propose all work by computing images of a signature revealing vector 
mod a series of primes,
and construction of the integer vector via the Chinese Remainder Algorithm.
%An integer answer such as characteristic polynomial and
%minimal polynomial of an integer matrix is often done by modular methods.
%It works as follows: first compute the answer mod many different primes, 
%then use the Chinese remainder algorithm to reconstruct the integer answer,
%or lift the answer to a power of a single prime, then reconstruct the answer
%by its $p$-adic representation.
If a rational answer is desired,
a rational reconstruction can be used to reveal the answer at the final step.
In this paper, we are interested in computing the characteristic polynomial,
minimal polynomial and leading principal minors of an integer matrix, all
of them can be represented as a vector of integers. 
Next, a generic Chinese remainder algorithm for such vectors 
with an early termination technique is described.
\begin{algorithm} {GenCRA}\\
$[$Chinese Remainder Algorithm with early termination] 
\Inspec
\begin{itemize}
\item
  $A$, a symmetric integer matrix.
\item
  $\reveal(v, A, p)$, a function takes two inputs $A$, $p$, and compute
  integer answer mod $p$ and store in in-out vector $v$.
  This function will return false if \reveal can't compute integer answer mod $p$ correctly,
  or true if it computes the integer answer mod $p$. 
\item
  $\set$, a set of primes of size $\pSize$, with minimal prime $\minPrime$.
\item
  $\rSize$, the random sample size.
\end{itemize}
\Outspec\\
The revealed vector $v$ with integer coefficients. \\
%\Stmt[1.]
%$r := -1$, initially guess rank is $0$\\
\Stmt[1.]
set $n := 0$, the length of output vector.\\
set list $\List := \emptyset$, list of pair (good prime, revealed answer
 mod that prime).\\
\Stmt[2.]
choose a random vector $x$ of length $n$ with
entries independently and uniformly chosen 
from $[0, \rSize - 1]$.\\
\Stmt[3.]
Uniformly and randomly choose a prime $p$ from $\set$,
until $p$ is not used before. 
Mark $p$ as being used.
call $\reveal(v_p, A, p)$.\\ 
If return value is false, goto statement 3.\\
If $v_p$'s length is lower than $n$, goto statement 3.\\
If $v_p$'s length is larger than $n$, set $n$ to be this length,
reset $\List$ to be the single pair ($p$, $v_p$),
and goto statement 2.\\
If the length of $v_p$ is equal to $n$, append pair ($p$, $v_p$) to $\List$.
Use Chinese remainder algorithm to construct 
a certificate $c^{(i)}$ where $i$ the size of $\List$, such that
$c^{(i)} = x \cdot v_{q} \pmod {q}$, for each pair $(q, v_q)$ in $\List$.\\
\Stmt[4.]
If $i = 1$ or $c^{(i)} \not= c^{(i-1)}$, then goto statement 3.
Otherwise $c^{(i)} = c^{(i-1)}$,
then the termination condition is met,
return the vector $v$, which is
constructed from pairs in $\List$ by the Chinese remainder algorithm,
such that $v = v_q \pmod q$, for every pair ($q, v_q$) in $\List$.
This construction can be done by using a divide-and-conquer method. 
%Quit gathering images when the termination condition is met, and 
%The Termination condition may be 
%deterministic, namely modulus greater than Hadamard bound bound,
%in which case the overall algorithm is Las Vegas.
%The Termination condition may be 
%probabilistic, namely stop by Zhendongs trick (fixme - to be explained here),
%in which case the overall algorithm is Monte Carlo.
\end{algorithm}
Note:
1. The Chinese remainder algorithm can be used to construct a number $a$ from
its remainders $r_1, \ldots, r_n$ moduli a bunch of primes $p_1, \ldots, p_n$
,respectively. We need to normalize the final number $a$, 
such that $a$ lies between $[-m/2, m/2]$, 
where $m$=$\prod_{1\leq i \leq n} p_i$.\\
2. An early termination technique is used here to speedup the computation.
There is already some discussion about the early termination 
-see e.g. \cite{Emiris98}.
Here, we use a different and more efficient termination technique.
At each step, only one integer called ``certificate" is constructed 
instead of the actual answer mod modulus so far, and has almost
the same probability. This technique is easy to adopt to other cases,
such as solving a non-singular integer systems over the rationals.\\
3. More than one independently certificates can reduce the error probability.
Since the error for each independently certificate are independent.
In practice, we use two independently certificates.
\begin{theorem}
On the condition that the length of output is correct,
the probability that output $v$ is not correct
is at most
$\frac{2}{\rSize} +$
$\frac{\log^2(2\length \rSize B)}
{\log^2(\minPrime) (\pSize - \pSize^\prime - \log(2\length \rSize B) / \log(\minPrime))}$,
where $\length$ is the length of the correct answer,
$B$ is infinity norm of the correct answer,
$\pSize^\prime$, number of primes $p$, 
such that either \reveal($v_p, A, p$) returns false
or $v_p$ has length of less than $\length$.
\end{theorem}

Note: For a typical problem, such as characteristic polynomial, 
minimal polynomial, it is easy to choose a reasonable set of prime 
and random sample size,
such that the error probability is very tiny like one in a million.

PROOF:
Let $\alpha$ denote the correct answer and
$c$ denote $x \cdot \alpha$. We assume early termination condition meets when 
$c^{(n)}$ is equal to $c^{(n-1)}$, for some number $n$. 
Obviously, if both $\abs {c} >= \normoo {\alpha}$ and
$c = c^{(n)}$ are true, then the algorithm returns the correct answer.

%Let $\alpha$ be the correct answer, 
%and $c = x \cdot \alpha$.
%Under assumption, $\alpha$ is a vector of length $\length$.
%Also Let pair ($p_i, v_{p_i}$)  in the $\List$ in order.
%We define $\normoo {t}$ to be  the maximal absolute value of entries of $x$,
%for any vector $t$. 
%%We call a prime $p$ ``a good prime" if $\reveal(v, A, p)$
%returns true and the length of output vector $v$
%%equals $\length$, otherwise we call it
%``a bad prime".
%
%Claim 1:If $x$ is a random vector with entries 
%independently and uniformly chosen from the integer set of $[0, \rSize-1]$,
%then $\Prob (\abs(x \cdot \alpha) < \normoo{\alpha})$
%$\leq \frac{2}{\rSize}$.
%This claim is true since
%there is at least one entry of $\alpha$, whose absolute value is
%equal to $\normoo{\alpha}$, and
%if with loss of generosity, we assume $\abs {\alpha_0}$ = $\normoo{\alpha}$,
%then any $x_1, \ldots, x_{\length-1}$, then there is at most two integers $x_0$, 
%such that $\abs{x \cdot \alpha} < \normoo{\alpha}$.
%
%Claim 2: For an integer $m$, if a random prime $p$ is uniformly chosen 
%from a cardinality $k$ subset of $\set$, then
%$\Prob(p | m) \leq \frac{\log(\abs{m})}{\log(\minPrime)k}$.
%This claim is true since there is at most $\frac{\log(\abs{m})}{\log(\minPrime)}$
%primes which are members of $\set$ and also divide $m$.
%
%When the early termination first meets at $c^{(n)} = c^{(n-1)}$,
%let us look at the probability of $c \not= c^{(n)}$.
%We have $c^{(n)} = c^{(n-1)}$, that means $c - c^{(n-1)} = 0 \pmod{p_n}$,
%and moreover, $\normoo {c - c^{(n-1)}} \leq 2\normoo{c}
%\leq 2 \length \rSize B$.
%
%For any fixed $n$, if $c \not= c^{(n-1)}$ and 
%a good prime $p_n$ is randomly uniformly chosen from $\set$,
%then $\Prob (p_n~|~(c - c^{(n-1)}))
%\leq \frac{\log_2(2\length \rSize B)}{\log(\minPrime) 
%(\pSize - \pSize^\prime - n)}$. 
%If $n$ is larger than
%$\frac{\log(2 \length \rSize B)}{\log(\minPrime)}$, 
%then $v$ must be equal to $c$.
%Since the constructed $v = c \pmod m$, and 
%$m = \prod_{1 \leq i \leq n}p_i > 2\normoo{c}$, 
%
%So when the early termination condition meets,
%$\Prob (c \not= c_{i-1}) \leq 
%\frac{(\log(2\length \rSize)^2}{(\log(\minPrime))^2 
%%(\pSize - \pSize^\prime - {\log(2 \length \rSize B)}/{\log(\minPrime)})}$.

%If both $\normoo{c} \geq \normoo{\alpha}$ and $c = c_n$ are true, then
%$v$ must be equal to the correct answer.
%So the  probability of
%$v \not= \alpha$ is
%at most
%$\frac{1}{\rSize}+$
%$\frac{\log^2(2\length \rSize B)^2}
%{\log^2(\minPrime) 
%(\pSize - \pSize^\prime - {\log(2 \length \rSize B)}/{\log(\minPrime)})}$.
\QED
