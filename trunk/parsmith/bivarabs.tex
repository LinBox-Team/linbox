\documentstyle{article}
\setlength{\textwidth}{6.5in}
\setlength{\textheight}{9in}
\setlength{\oddsidemargin}{0pt}
\setlength{\evensidemargin}{0pt}
\setlength{\topmargin}{-.5in}

\begin{document}

\begin{center}
\large{\bf Parametric Linear Systems:\\
the two parameter case} \\[10pt]
Mark Giesbrecht\\
Department of Computer Science\\
University of Manitoba\\
Winnipeg, MB, R3T 2N2, Canada\\
\verb|mwg@cs.umanitoba.ca|\\[5pt]
David Saunders \\
Department of Computer and Information Sciences \\
University of Delaware \\
Newark, DE 19716 USA\\
\verb|saunders@udel.edu|
\end{center}

\newcommand{\F}{{\mathsf F}}
\newcommand{\ints}{{\mathbf Z}}
\newcommand{\mxn}{{m\times n}}
\newcommand{\nx}[1]{{n\times #1}}
\newcommand{\mx}[1]{{m\times #1}}
\newcommand{\rx}[1]{{r\times #1}}
\newcommand{\nequiv}{{\mskip4mu\not\equiv\mskip4mu}}
\def\nin{{\mathrel{\hbox{$\mskip4mu\not\in\mskip4mu$}}}}

When the coefficients of a system of linear equations are polynomial
expressions in parameters over a field, the solutions at values of
these parameters in the field may be expressed as a finite number of
cases (``regimes'' to use William Sit's terminology).  Each case
involves a polynomial condition on the parameters and a description of
the solution space which involves the parameters and is applicable for
those values of the parameters which satisfy the condition.

We offer an algorithm which produces in polynomial time a complete set
of solutions when the coefficients are polynomials in two parameters.
Specifically, given a matrix $A(x,y)\in\F[x,y]^\mxn$ for a field $\F$
and a vector $b(x,y)\in\F[x,y]^\mx1$, we show how to describe all of
the solutions $u \in \F^\nx1$ to all of the systems
$\{ A(\alpha, \beta)u = b(\alpha, \beta):\; \alpha,\beta\in\F\}$.
%
The description consists of a finite sequence of semi-algebraic sets
$C_1,C_2,\ldots\subseteq\F^2$, where
$
C_i = \left\{ (\alpha,\beta)\in P_i ~\hbox{and}~ 
              (\alpha,\beta)\nin Q_i \right\}
$
and $P_i$ and $Q_i$ are algebraic sets, each described as the variety
of an ideal generated by a short list of polynomials in $\F[x,y]$.
For each $C_i$ we provide a parametric description of the solution
space on this set by
$u = u_0(x,y) + V(x,y)w$, where $u_0(x,y) \in \F[x,y]^\nx1$ is a particular
solution, the columns of $V(x,y) \in \F[x,y]^{n\times l}$ are a basis
for the nullspace of $A$ under the given condition, and $w$ is
arbitrary in $\F^{l\times 1}$. 
%
We give bounds on the number and sizes of these solution regimes
which are polynomial in the matrix size and in the degrees of the
entries in the parameters $x$ and $y$.  Similarly, the number of field
operations in $\F$ is polynomially bounded.  For many familiar fields
such as the rational numbers, the bit operation costs are also
polynomially bounded.
 
Our approach is to compute a Smith form of $A(x,y)$ viewed as a matrix
over the principal ideal domain $\F(y)[x]$.  This gives a sequence of
general solutions defined by the Smith form entries, which however are
subject to finitely many exceptions described by a condition 
$f(y) = 0$ for an $f\in\F[y]$.  We then show how to carry out Smith form
computations in $(\F[y]/(f))[x]$ to solve the exceptional cases.  We
remark that the one parameter case is solved by a single Smith form
computation and that the $t$ parameter case requires exponential time
in general.  In fact there can be a number of solution regimes which
is exponential in $t$.

A consequence of our method is that we can determine the variety of
each determinantal ideal and a basis for the radical of each
determinantal ideal.  Recall that the $k$-th determinantal ideal of
$A$ is the ideal in $\F[x,y]$ generated by all the $k \times k$ minors
of $A$.  In particular we determine the McCoy rank of $A$, which is
the greatest $k$ such that the determinantal ideal contains $1$, and
we determine the generic rank, the greatest $k$ such that the $k$-th
determinantal ideal is not $(0)$.

\end{document}


